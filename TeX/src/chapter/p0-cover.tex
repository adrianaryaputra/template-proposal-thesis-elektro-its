\docDefaultSpacing

\begin{titlepage}

    \pagenumbering{roman}
    \setcounter{page}{1}
    
    \thispagestyle{empty}

    % LAMBANG ITS
    \begin{tikzpicture}[overlay,remember picture] 
        \node[anchor=north west, shift={(64mm, 5mm)}] at (current page.north west){\includegraphics[width=35mm]{src/image/its-lambang.png}}; 
    \end{tikzpicture}
    
    % GARIS KUNING
    \begin{tikzpicture}[overlay,remember picture] 
        \node[anchor=north west, shift={(24mm, -45mm)}] at (current page.north west){\includegraphics[width=210mm]{src/image/garis-kuning.png}}; 
    \end{tikzpicture}

    % TEKS SAMPUL
    \begin{flushleft}
        \vspace{\coverSpacing}
        \textbf{PROPOSAL THESIS}
    \end{flushleft}

    \vspace{-2mm}\Large
    \begin{flushleft}\textbf{\MakeUppercase{\judulThesis}}\end{flushleft}

    \vspace{10mm}
    \begin{flushleft}
        \normalsize
        \MakeUppercase{\mahasiswaNama} \\
        \MakeUppercase{\mahasiswaNRP} \\
        \vspace{1em}
        DOSEN PEMBIMBING \\
        \pembimbingSatuNama \\
        \pembimbingDuaNama \\
        \vspace{1em}
        PROGRAM MAGISTER \\
        BIDANG KEAHLIAN \MakeUppercase{\bidangKeahlian} \\
        DEPARTEMEN \MakeUppercase{\departemen} \\
        FAKULTAS \MakeUppercase{\fakultas} \\
        INSTITUT TEKNOLOGI SEPULUH NOPEMBER \\
        SURABAYA \\
        \tahunPenulisan
    \end{flushleft}

    
\end{titlepage}

% \restoregeometry{}