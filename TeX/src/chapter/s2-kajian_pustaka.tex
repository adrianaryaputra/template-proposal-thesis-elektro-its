\chapter{Kajian Pusataka}{
    \lipsum[4]
}


\section{Kajian Penelitian Terkait}{
    \lipsum[6]
    \begin{f_enumerate}
        \item contoh item satu, boleh panjang boleh pendek. contoh item satu, boleh panjang boleh pendek. contoh item satu, boleh panjang boleh pendek. contoh item satu, boleh panjang boleh pendek.
        \item contoh item satu, boleh panjang boleh pendek. contoh item satu, boleh panjang boleh pendek. contoh item satu, boleh panjang boleh pendek. contoh item satu, boleh panjang boleh pendek.
        \item contoh item satu, boleh panjang boleh pendek. contoh item satu, boleh panjang boleh pendek. contoh item satu, boleh panjang boleh pendek. contoh item satu, boleh panjang boleh pendek.
    \end{f_enumerate}
    \lipsum[2]
}


\section{Teori Dasar}{
    \lipsum[9-11]
}

\subsection{Cara Membuat Gambar}{
    Untuk menambahkan gambar, simpan gambar dalam folder ./src/image/. Gambar akan diletakkan secara otomatis di tempat yang paling sesuai. cukup tulis perintah latex berikut. setelah itu gambar dapat direferensikan dengan cara menuliskannya seperti \refgbr{fig: lambang its}.
    % SATU GAMBAR DEGAN LABEL 
    \begin{figure}
        \centering
        \includegraphics[width=.3\textwidth]{src/image/its-lambang.png}
        \caption{Lambang ITS}
        \label{fig: lambang its}
    \end{figure}

    Beberapa gambar dapat ditampilkan secara bersamaan menggunakan subfigure. dan dapat dikutip sebagian, seperti \refgbr{fig:lambang its a}, atau bisa juga hanya diambil penomoran subfigurnya seperti \ref{sub@fig:lambang its b}, atau figurnya secara keseluruhan \refgbr{fig:lambang its semua}.
    \begin{figure}
        \centering
        \begin{subfigure}[b]{0.3\textwidth}
            \centering
            \includegraphics[width=\textwidth]{src/image/its-lambang.png}
            \caption{Lambang ITS}
            \label{fig:lambang its a}
        \end{subfigure}
        \begin{subfigure}[b]{0.3\textwidth}
            \centering
            \includegraphics[width=\textwidth]{src/image/its-lambang.png}
            \caption{Lambang ITS juga}
            \label{fig:lambang its b}
        \end{subfigure}
        \begin{subfigure}[b]{0.3\textwidth}
            \centering
            \includegraphics[width=\textwidth]{src/image/its-lambang.png}
            \caption{Lambang ITS lagi}
            \label{fig:lambang its c}
        \end{subfigure}
        \caption{Lambang-Lambang ITS}
        \label{fig:lambang its semua}
    \end{figure}    
}

\subsection{Cara Membuat Tabel}{
    Di segmen ini akan ditampilkan berbagai macam cara untuk membuat tabel. Untuk membuat tabel kita dapat menggunakan script dibawah ini. 
    \begin{table}[h]
        \centering
        \begin{tabular}{| l | l | l |}
            \hline
            A & B & C \\
            \hline
            1 & 2 & 3 \\
            4 & 5 & 6 \\
            \hline
        \end{tabular}
        \caption{Tabel sangat sederhana}
        \label{tab:abc}
    \end{table}

    \begin{table}[h]
        \begin{subtable}[h]{0.45\textwidth}
            \centering
            \begin{tabular}{l | l | l}
            Day & Max Temp & Min Temp \\
            \hline \hline
            Mon & 20 & 13\\
            Tue & 22 & 14\\
            Wed & 23 & 12\\
            Thurs & 25 & 13\\
            Fri & 18 & 7\\
            Sat & 15 & 13\\
            Sun & 20 & 13
           \end{tabular}
           \caption{First Week}
           \label{tab:week1}
        \end{subtable}
        \hfill
        \begin{subtable}[h]{0.45\textwidth}
            \centering
            \begin{tabular}{l | l | l}
            Day & Max Temp & Min Temp \\
            \hline \hline
            Mon & 17 & 11\\
            Tue & 16 & 10\\
            Wed & 14 & 8\\
            Thurs & 12 & 5\\
            Fri & 15 & 7\\
            Sat & 16 & 12\\
            Sun & 15 & 9
            \end{tabular}
            \caption{Second Week}
            \label{tab:week2}
         \end{subtable}
         \caption{Suhu max dan min di bulan Juni}
         \label{tab:temps}
    \end{table}
}

\subsection{Cara Melakukan sitasi}{
    Sitasi dapat dilakukan menggunakan command \cite{han_heo_park_kee_sunwoo_2016}.
    \lipsum[2]
}

\subsection{Theory 1}{
    \lipsum[6-8]
}

\subsection{Theory 2}{
    \lipsum[6-8]
}